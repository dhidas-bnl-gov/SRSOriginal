\documentclass{article}

\title{SRS Python User Interface}
\author{Dean Andrew Hidas}
\date{\today}


\begin{document}

\maketitle

\begin{abstract}
This document describes the python interface to SRS.
\end{abstract}

\section{Introduciton}
The SRS software package is intended to calculate trajectories of charged particles in magnetic fields and the radiative properties of such scenarios.  Other than numerical discritization, no approximations are made in these calculations and they are valid for relativistic and non-relativistic


\section{Conventions}
\begin{itemize}
\item \begin{verbatim}[x, y, z]\end{verbatim}
\item \begin{verbatim}[x, y, z, t], [Px, Py, Pz, E]\end{verbatim}
\end{itemize}


\section{Basic Examples}
In order to run a basic example simulation you must first understand roughly what this software is doing at what it requires as user input.  Please read the XXX.


\section{Philosophy}


\section{Large Scale Data}


\section{Member Functions}

\end{document}
